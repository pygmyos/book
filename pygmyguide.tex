\documentclass{article}

\usepackage[T1]{fontenc}
\usepackage{acronym}
\usepackage{verbatim}
\usepackage{color}
\usepackage{listings}

\definecolor{darkgreen}{rgb}{0,0.6,0}

% listings environment for code
\lstset{ %
	language=C,						% code language
	basicstyle=\footnotesize,       % font size
	numbers=left,                   % line number location
	numberstyle=\footnotesize,      % line number font size
	stepnumber=1,                   % line number step
	numbersep=5pt,                  % margin between line numbers and code
	backgroundcolor=\color{white},  % background color
	showspaces=false,               % visual spaces
	showstringspaces=false,         % visual string spaces
	showtabs=false,                 % visual string tabs
	frame=none,						% frame code
	tabsize=2,						% size of tabs
	captionpos=b,					% caption position (b=bottom)
	breaklines=true,				% automatic line breaking
	breakatwhitespace=true,			% automatic line breaking only at whitespace
	keywordstyle=\color{blue},		% key word color
	commentstyle=\color{darkgreen},	% comment color
	stringstyle=\color{blue},		% string literal color
	escapechar=\$
}

% These are used for code segments
\newcommand{\dq}{\textquotedbl}
\newcommand{\sq}{\textquotesingle}

\begin{document}
\title{A Guide to PygmyOS}
\author{Warren D Greenway \and Evan Teitelman}
\maketitle

\tableofcontents

\section{Acronyms to Know}
\begin{acronym}
\acro{PWM}{Pulse-Width Modulation}
\acro{GPIO}{General Purpose In/Out}
\end{acronym}

\section{What is PygmyOS?}
PygmyOS is a simple, small, yet scalable operating system for embedded systems. Its primary goals are to be clean, simple, and efficient. Pygmy is written in boiler-plate C for the sake of efficiency and simplicity. Pygmy also follows simple naming conventions and allows the OS to be integrated with the end application in a staged manner, from simply wrapping the hardware in a user friendly set of APIs, to providing advanced tasking, clock, utility, command processing, file system, graphical and messaging services.

At the end of this document you will find an API reference. In the following chapters I'll be including a basic overview of the API as it applies to the examples presented. Currently, I have no intention of including source code with these tutorials. From a practical perspective, you don't learn when you copy and paste. You need to drill the basics in. With that thought in mind, however, I'll be relying on brevity to avoid applying too much load to your patience and typing skills.

Let's start with a basic description of your \verb|main.c| source file.

\begin{lstlisting}
#include $\dq$pygmy_profile.h$\dq$

void main(void)
{
	while(1){ ; }
}
\end{lstlisting}

So, let's start at the top. For each Pygmy project there should be a file named \verb|pygmy_profile.h| that contains the profile information for your specific platform, or hardware configuration. The contents of \verb|pygmy_profile.h| will look something like this:

\begin{lstlisting}
#pragma once
#include $\dq$pygmy_sys.h$\dq$
#include $\dq$pygmy_port.h$\dq$
#include $\dq$pygmy_com.h$\dq$
#include $\dq$pygmy_audio.h$\dq$
#include $\dq$pygmy_adc.h$\dq$
#include $\dq$pygmy_file.h$\dq$
#include $\dq$pygmy_lcd.h$\dq$
#include $\dq$pygmy_gui.h$\dq$
#include $\dq$pygmy_nvic.h$\dq$
#include $\dq$pygmy_rtc.h$\dq$
#include $\dq$pygmy_string.h$\dq$
#include $\dq$pygmy_web.h$\dq$
#include $\dq$profiles/pygmy_nebula.h$\dq$
\end{lstlisting}

The first line is for the compiler's pre-processor only, it keeps the header file \verb|pygmy_profile.h| from being included multiple times. The next group is the Pygmy API includes. The only file included in the API sources is \verb|pygmy_profile.h|, which means it is critical to your project, and that every API used must be listed. It may be useful to note that PygmyOS is designed to be compiled with size optimization and including all APIs will not cause a decrease in performance or an increase in code size. In light of this fact, it is generally a good idea to include all the APIs to avoid complications with missing resources as your project evolves.

The last line is the most interesting. In this case the included file is the standard hardware definition for the the Pygmy Nebula board. If a shield or backpack module or modules are to be used with the main board, they will be included just below the primary board include file. All Pygmy profiles for modules and platforms (main boards) are stored in the folder \verb|pygmy/profiles/|. The file \verb|pygmy_profile.h| must be in the same folder as \verb|main.c|. One other note on \verb|pygmy_profile.h| is that it can be used as the header file for \verb|main.c|.

Let's get back to our \verb|main.c| and finish our analysis. The main function is void and void. No parameters are passed to main or returned by main. Dropping out of main will return to the startup function and an endless loop. Despite this fact, it is still recommended to insert an infinite loop at the end of main. Generally nothing will be executed in the loop.

Now that we have gone over a basic main source file, we look at a few examples.

\section{Getting the Blink}

Goal: Blink an LED

Platform: Pygmy Nebula

API description:

\begin{lstlisting}
pygmy_port
void pinConfig(u8 ucPin, u8 ucMode)
void pinSet(u8 ucPin, u8 ucState)
void pinGet(u8 ucPin)
pygmy_sys
u16 sysInit(void)
u16 newSimpleTask(u8 *ucName, u32 ulTimer, PYGMYFUNCPTR EventHandler)
\end{lstlisting}

It's a good idea to have balance before taking a step. It's a good idea to learn to walk before you try to run. In line with this reasoning, we'll start by turning an LED on, LED0 to be specific. LED0 and LED1 are the two user LEDs on the Pygmy Nebula Board. By setting the pin to OUTPUT and setting the pin HIGH, the LED will power on.

\begin{lstlisting}
#include $\dq$pygmy_profile.h$\dq$

void main(void)
{
	pinConfig(LED0, OUT);
	pinSet(LED0, HIGH);
	while(1){ ; }
}
\end{lstlisting}

Loading and booting this simple program will light up LED0 on the board and wait in the loop. As you can see, the program is incredibly simple. Making the LED blink will be our next task. There are many ways to accomplish this, all reasonably simple. In the example we will use the tasking system to blink the LED. This approach will be much like driving a tack with a sledge hammer. There is a reason for using this specific approach, we will be demonstrating with this simple example how to use the timing and tasking systems. Once you start coding for Pygmy, you'll quickly find the tasking system to be a powerful tool to solve a wide variety of problems.

\begin{lstlisting}
#include $\dq$pygmy_profile.h$\dq$
void blink(void);
void main(void)
{
	sysInit();
	pinConfig(LED0, OUT);
	newSimpleTask($\dq$blink$\dq$, 100, blink);
	while(1){ ; }
}

void blink(void)
{
	if(pinGet(LED0))
	{
		pinSet(LED0, LOW);
	}
	else
	{
		pinSet(LED0, HIGH);
	}
}
\end{lstlisting}

Got that? You now have a task (thread) running in the background. First, \verb|sysInit()| initializes the task and messaging systems. Second, \verb|pinConfig()| is called to set \emph{LED0} to output. Third, \verb|newSimpleTask()| is called to spawn a thread named \verb|blink|, with an interval of 100 milliseconds; the function blink() will execute every 100 milliseconds.

When blink is executed, \verb|pinGet()| returns the current state of \emph{LED0}, and toggles the pin state. The task will execute indefinitely and requires very little resources to execute. You can of course start multiple tasks. You can blink \emph{LED1} in opposition to \emph{LED0} in this way.

I'll provide another LED blinking example below that alternates the two user LEDs using two threads (yes, it obviously could be done with the one thread). Note in this case that instead of using a static variable to store the state of the LEDs, we are using \verb|pinGet()|. Due to this fact, we must \verb|setPin()| one of the LEDs LOW to start them out opposite to eachother. Why? Simply because the port pins default to HIGH.

\begin{lstlisting}
#include $\dq$pygmy_profile.h$\dq$

void blink1(void);
void blink2(void);

void main(void)
{
	sysInit();
	pinConfig(LED0, OUT);
	pinConfig(LED1, OUT);
	pinSet(LED1, LOW);
	newSimpleTask($\dq$blink1$\dq$, 100, blink1);
	newSimpleTask($\dq$blink2$\dq$, 100, blink2);

	while(1){ ; }
}

void blink1(void)
{
	if(pinGet(LED0))
	{
		pinSet(LED0, LOW);
	}
	else
	{
		pinSet(LED0, HIGH);
	}
}

void blink2(void)
{
	if(pinGet(LED1))
	{
		pinSet(LED1, LOW);
	}
	else
	{
		pinSet(LED1, HIGH);
	}
}
\end{lstlisting}

\section{Pulse-Width Modification}

Goal: Blink an LED

Platform: Pygmy Nebula

API description:

\begin{lstlisting}
pygmy_port
u8 pinPWM(u8 ucPin, u32 uiFreq, u8 ucDutyCycle)
pygmy_sys
u16 sysInit(void)
u16 newSimpleTask(u8 *ucName, u32 ulTimer, PYGMYFUNCPTR EventHandler)
\end{lstlisting}

Now that you have LEDs blinking, it's time to take the next step. Our next goal is to have \emph{LED0} flash instead of simply blink. By this, I mean that we will linearly increase and then decrease the light intensity. How will we accomplish this goal? Simple. We will use \ac{PWM} to vary the average DC voltage.

With \ac{PWM} you are basically switching the power from high to low at a given frequency for a specific percentage of the time to vary the average voltage. Fifty percent duty cycle would result in approximately fifty percent of the supply voltage at the pin being toggled.

A real description of \ac{PWM} is beyond the scope of this document. However, if you plan on using \ac{PWM} for partial voltage generation or hobby servo positioning, you should go read up on Wikipedia at the least.

Pygmy makes \ac{PWM} simple. All you have to do to set \ac{PWM} on a timer pin is: \verb|pinPWM(pin, frequency, dutycyle);|. The pin will be configured to the correct mode and the peripheral clock sources configured and the timers configured automatically. The \ac{PWM} will continue on the pin or pins selected until the pin mode is changed or the timer disabled. Even if every timer output is setup as \ac{PWM} and actively being driven, there is no load on the CPU caused by this activity.

Before we look at sample code, it is high time to review the Pygmy pin naming convention. This is simple in arrangement and will only take a moment to review.
\begin{description}
	\item[Digital IO] $D_0$, $D_1$, $D_3$ \ldots $D_x$
	\item[Analog Input / Digital IO] $A_0$, $A_1$, $A_2$ \ldots $D_x$
	\item[Timer Output / Digital IO] $T_0$, $T_1$, $T_2$ \ldots $T_x$
	\item[Analog Input / Timer Output / Digital IO] $TA_0$, $TA_1$, $TA_2$ \ldots $TA_x$
\end{description}

Using this convention, the pin name tells you what limitations the pin has for usage. The pin name on the board is the same name you will use in the function calls, or alternately the \ac{GPIO} name for the pin (\emph{PA11} for \emph{LED0}, as an example). Additional functions may exist on the pin in question; review the ST Reference for the MCU in question for additional details if the standard behavior isn't what you're looking for.

Besides the generic IO pins, there are also special purpose pins that may also be used as general purpose IO if desired. Some examples of these special purpose pins are as follows:
\begin{description}
	\item[COM1\_TXTransmit] COM1 serial port
	\item[COM1\_RXReceive] COM1 serial port
	\item[COM2\_TXTransmit] COM2 serial port
	\item[COM2\_RXReceive]  COM2 serial port
	\item[COM3\_TXTransmit] COM3 serial port
	\item[COM3\_RXReceive] COM3 serial port
	\item[DAC1] Digital to Analog Converter Channel 1
	\item[DAC2] Digital to Analog Converter Channel 2
	\item[MCO] Master Clock Out
\end{description}

Now that we have discussed the basics of pin naming, we may continue with our code example of \ac{PWM}. Please note that \emph{LED0} and \emph{LED1} are significantly different electrically in that \emph{LED0} is connected to \ac{GPIO} \emph{PA11} while LED1 is connected to \ac{GPIO} \emph{PA12}. You will find in the reference manual for the MCU (should you have the desire to read said manual) that \emph{PA11} is connected to Timer1 Channel4, while \emph{PA12} is not connected to a timer output at all. This means that the following sample code will only work with \emph{LED0}. \emph{LED1} is strictly a digital IO.

\begin{lstlisting}
#include $\dq$pygmy_profile.h$\dq$

void flash(void);

void main(void)
{
	sysInit();
	newSimpleTask($\dq$flash$\dq$, 100, flash);
	while(1){ ; }
}

void flash(void)
{
	static s8 i = 0, cStep = 1;
	i += cStep;
	if(i > 80)
	{
		cStep = -1;
	}

	if(i < 1)
	{
		cStep = 1;
	}
	pinPWM(LED0, 8000, i);
}
\end{lstlisting}

\section{COM Ports}
Goal: Send and Receive data using COM3
Platform: Pygmy Nebula
API description:
\begin{lstlisting}
pygmy_com
void comConfig(u8 ucPort, u8 ucProtocol, u8 ucOptions, u32 uiRate)
pygmy_sys
void streamInit(void)
u8 streamSetRXBuffer(u8 ucStream, u8 *ucBuffer, u16 uiLen)
u8 streamSetPut(u8 ucStream, void *ptrFunc)
u8 streamSetGet(u8 ucStream, void *ptrFunc)
u8 streamGetChar(u8 ucStream)
void print(u8 ucStream, u8 *ucBuffer, ...)
\end{lstlisting}

LEDs are all well and fine, but once you're done blinking the user LEDs every way you can imagine, you'll likely want to move on to something more useful. Being able to communicate through the COM ports allows you to dump debug info, upload or download files to memory, read sensor values, or implement a full-featured command interface. The good news is that serial communication is a breeze with Pygmy.

An important note here is that the Nebula boards running the \emph{F103} and \emph{F100} series MCUs do not have native USB. What this means in practical terms is that communications through the USB port on the Nebula boards is accomplished via the serial interface. Internally, \emph{COM3} is connected to a USB transceiver IC (MCP2200) which accomplishes the translation to USB in hardware. So, for these tutorials we'll be focusing on the serial communication interface using COM3.

Our first foray into serial communications will cover configuring the port and sending a string. There are low level and high level interfaces available in Pygmy, but we'll be focused on the high level interface.

\begin{lstlisting}
#include $\dq$pygmy_profile.h$\dq$

void echoCOM3(void);

void main(void)
{
	streamInit();
	streamSetPut(COM3, putsUSART3);
	print(COM3, $\dq$Hello World$\dq$);
	while(1){ ; }
}
\end{lstlisting}

Let's start near the beginning and work our way back. The first function called, \verb|streamInit()| configures the stream interface used by \verb|print()| to defaults. The next function, \verb|streamSetPut()|, selects the interface for the selected COM interface (in this case \emph{COM3}). The second parameter is the function for output. The function for output can be user defined or one of the system provided defaults. In this case we are connecting the stream output for \emph{COM3} to \verb|putsUSART3()|, which is a basic output interface for \emph{USART3 TX}.

The final function call is the meat of the program. A call to \verb|print()| with \emph{COM3} selected as the stream and a string to print will cause \verb|Hello World| to be printed out the USB port. To see this, use a terminal program such as Minicom for Unix-like operating systems or TeraTerm for Windows\textsuperscript{\textregistered}.

Now we must back track to what happens before the first function call. If you were especially observant, you may have noticed that we have yet to setup clocks, or, in this case, even the USARTs. The missing link is the bootloader.

Before transfering control to the user program, the bootloader configures \emph{USART3}, the watchdog, the clocks, and SysTick timer. Since we are using \emph{USART3} connected to USB, we don't have to configure the port (or any other resources configured by the bootloader). The exception would be if we needed to change the baud rate, in which case we would call \verb|comConfig()| as in the example below. Please note that in this example code \verb|comConfig()| is being called to demonstrate usage, but isn't actually required.

\begin{lstlisting}
#include $\dq$pygmy_profile.h$\dq$
void comtest(void);

u8 globalBuffer[ 128 ];

void main(void)
{
	streamInit();
	streamSetRXBuffer(COM3, globalBuffer, 128);
	streamSetPut(COM3, putsUSART3);
	streamSetGet(COM3, comtest);
	comConfig(COM3, 0, 0, 230400);

	while(1){ ; }
}

void comtest(void)
{
	print(COM3, $\dq$%c$\dq$, streamGetChar(COM3));
}
\end{lstlisting}

Notice the use of \verb|comConfig()| in this iteration. We have also added a function to handler received characters. There are many ways to customize the comports, but one of the most common will be a modified receive handler. In this case we are allowing Pygmy to receive and buffer the incoming chars in the buffer we allocated. Pygmy will call our custom handler \verb|comtest()| each time a char is received.

Since this is nothing more than a demonstration, we are merely echoing the character back to the originating comport.
\end{document}
